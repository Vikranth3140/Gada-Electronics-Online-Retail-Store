\documentclass[12pt]{article}
\usepackage{geometry}
\usepackage{enumitem}
\usepackage{graphicx}
\usepackage{hyperref}

\geometry{a4paper, margin=0.5in}

\begin{document}

\title{\textbf{DBMS Project\\ Gada Electronics Online Retail Store}}
\author{Group 10 \\ Vikranth Udandarao - 22570, Tharun Harish - 22541, \\ Hemanth Dindigallu - 22212, Aditya Prasad - 22036}
\date{}
\maketitle

\section*{SQL Queries}

\begin{verbatim}

use gada_electronics;

-- Retrieve all products with their respective quantities available in a specific warehouse by Pincode.
SELECT p.Product_ID, p.Name, p.Price, w.Warehouse_Quantity
FROM Product p
JOIN Warehouse w ON p.Product_ID = w.Product_ID
WHERE w.Pincode = '62704';


-- Update the quantity of a product in a customer's cart.
UPDATE Cart
SET Quantity = 2
WHERE Cart_ID = 'CART001' AND Product_ID = 'PROD001';


-- List customers who have made a purchase, along with their email and the total amount spent (consider discounts in calculations).
SELECT c.Customer_ID, c.Email, SUM((crt.Price * (1 - crt.Offer / 100)) * crt.Quantity) AS Total_Spent
FROM Customer c
JOIN Cart crt ON c.Customer_ID = crt.Customer_ID
JOIN Payment p ON crt.Cart_ID = p.Cart_ID
WHERE p.Status = 'Completed'
GROUP BY c.Customer_ID;


-- Find all products that have never been added to a cart (i.e., identify unsold products).
SELECT p.Product_ID, p.Name
FROM Product p
WHERE p.Product_ID NOT IN (SELECT DISTINCT Product_ID FROM Cart);


-- Show the total number of orders completed for each customer.
SELECT c.Customer_ID, c.Name, COUNT(o.Order_ID) AS Orders_Completed
FROM Customer c
JOIN Orders o ON c.Customer_ID = o.Customer_ID
JOIN Payment p ON o.Payment_ID = p.Payment_ID
WHERE p.Status = 'Completed'
GROUP BY c.Customer_ID;


-- Retrieve the list of products along with the names of customers who have them in their cart but have not completed payment.
SELECT p.Product_ID, p.Name AS Product_Name, c.Name AS Customer_Name
FROM Product p
JOIN Cart crt ON p.Product_ID = crt.Product_ID
JOIN Customer c ON crt.Customer_ID = c.Customer_ID
JOIN Payment pay ON crt.Cart_ID = pay.Cart_ID
WHERE pay.Status = 'Pending';


-- Increment the discount on all products by 5% up to a maximum of 20% discount.
UPDATE Product
SET Discount = LEAST(Discount + 5, 20);

-- Find the average price of products in each category (assuming a 'Category' field exists in the Product table).
SELECT Category, AVG(Price) AS Average_Price
FROM Product
GROUP BY Category;

-- List all customers and the number of addresses they have registered, including those with no registered address.
SELECT c.Customer_ID, c.Name, COUNT(a.Address_ID) AS Address_Count
FROM Customer c
LEFT JOIN Address a ON c.Customer_ID = a.Customer_ID
GROUP BY c.Customer_ID;


-- Delete all cart items for customers who have not logged in for over a year (assuming a 'Last_Login' timestamp field exists in the Customer table).
DELETE FROM Cart
WHERE Customer_ID IN (
  SELECT Customer_ID
  FROM Customer
  WHERE Last_Login <= DATE_SUB(CURDATE(), INTERVAL 1 YEAR)
);


\end{verbatim}


\section*{Relational Algebra}

\begin{verbatim}

\end{verbatim}


\section*{Constraints}

\begin{verbatim}

\end{verbatim}


\section*{Relational Schema}

\begin{verbatim}

\end{verbatim}


\section*{Contributions}
All the group members contributed equally to this submission.

\end{document}