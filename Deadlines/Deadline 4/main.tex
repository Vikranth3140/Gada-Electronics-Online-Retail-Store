\documentclass[12pt]{article}
\usepackage{geometry}
\usepackage{enumitem}
\usepackage{graphicx}
\usepackage{hyperref}

\geometry{a4paper, margin=0.5in}

\begin{document}

\title{\textbf{DBMS Project\\ Gada Electronics Online Retail Store}}
\author{Group 10 \\ Vikranth Udandarao - 22570, Tharun Harish - 22541, \\ Hemanth Dindigallu - 22212, Aditya Prasad - 22036}
\date{}
\maketitle

\section*{SQL Queries}

\begin{verbatim}

use gada_electronics;

-- Retrieve all products with their respective quantities available in a specific warehouse by Pincode.
SELECT p.Product_ID, p.Name, p.Price, w.Warehouse_Quantity
FROM Product p
JOIN Warehouse w ON p.Product_ID = w.Product_ID
WHERE w.Pincode = '62704';


-- Update the quantity of a product in a customer's cart.
UPDATE Cart
SET Quantity = 2
WHERE Cart_ID = 'CART001' AND Product_ID = 'PROD001';


-- List customers who have made a purchase, along with their email and the total amount spent (consider discounts in calculations).
SELECT c.Customer_ID, c.Email, SUM((crt.Price * (1 - crt.Offer / 100)) * crt.Quantity) AS Total_Spent
FROM Customer c
JOIN Cart crt ON c.Customer_ID = crt.Customer_ID
JOIN Payment p ON crt.Cart_ID = p.Cart_ID
WHERE p.Status = 'Completed'
GROUP BY c.Customer_ID;


-- Find all products that have never been added to a cart (i.e., identify unsold products).
SELECT p.Product_ID, p.Name
FROM Product p
WHERE p.Product_ID NOT IN (SELECT DISTINCT Product_ID FROM Cart);


-- Show the total number of orders completed for each customer.
SELECT c.Customer_ID, c.Name, COUNT(o.Order_ID) AS Orders_Completed
FROM Customer c
JOIN Orders o ON c.Customer_ID = o.Customer_ID
JOIN Payment p ON o.Payment_ID = p.Payment_ID
WHERE p.Status = 'Completed'
GROUP BY c.Customer_ID;


-- Increment the discount on all products by 5% up to a maximum of 20% discount.
UPDATE Product
SET Discount = LEAST(Discount + 5, 20);


-- List all customers and the number of addresses they have registered, including those with no registered address.
SELECT c.Customer_ID, c.Name, COUNT(a.Address_ID) AS Address_Count
FROM Customer c
LEFT JOIN Address a ON c.Customer_ID = a.Customer_ID
GROUP BY c.Customer_ID;


-- Show total sales per product, only listing those with sales exceeding 1,000 rupees.
SELECT p.Product_ID, p.Name, SUM(c.Quantity * (c.Price - (c.Price * c.Offer / 100))) AS TotalSales
FROM Product p
JOIN Cart c ON p.Product_ID = c.Product_ID
JOIN Payment pay ON c.Cart_ID = pay.Cart_ID AND pay.Status = 'Completed'
GROUP BY p.Product_ID
HAVING TotalSales > 1000;


-- Update stock quantities based on orders, ensuring the quantity does not drop below zero.
UPDATE Warehouse w
JOIN (
  SELECT crt.Product_ID, SUM(crt.Quantity) AS QuantitySold
  FROM Cart crt
  JOIN Payment pay ON crt.Cart_ID = pay.Cart_ID
  WHERE pay.Status = 'Completed'
  GROUP BY crt.Product_ID
) AS SoldItems ON w.Product_ID = SoldItems.Product_ID
SET w.Warehouse_Quantity = GREATEST(w.Warehouse_Quantity - SoldItems.QuantitySold, 0);


-- Display top 3 most frequently purchased products.
SELECT Product_ID, Name
FROM (
  SELECT p.Product_ID, p.Name,
    RANK() OVER (PARTITION BY p.Product_ID ORDER BY COUNT(c.Cart_ID) DESC) AS PurchaseRank
  FROM Product p
  JOIN Cart c ON p.Product_ID = c.Product_ID
  JOIN Payment pay ON c.Cart_ID = pay.Cart_ID AND pay.Status = 'Completed'
  GROUP BY p.Product_ID
) AS RankedProducts
WHERE PurchaseRank <= 3;


\end{verbatim}


\section*{Relational Algebra}

\begin{verbatim}

\end{verbatim}


\section*{Constraints}

\begin{verbatim}

\end{verbatim}


\section*{Relational Schema}

\begin{verbatim}

\end{verbatim}


\section*{Contributions}
All the group members contributed equally to this submission.

\end{document}